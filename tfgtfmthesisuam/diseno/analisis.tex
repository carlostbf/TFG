\paragraph{Requisitos funcionales}

\begin{functional}
        \item Permite definir el formato de los datos mediante un fichero Json.
    
        \item Permite  conocer las trayectorias de los terminales (mediante un algoritmo).
        \begin{functional}
        	\item Las trayectorias se mostrarán en un mapa.
        	\item Se descartarán las ubicaciones irrelevantes. (A concretar!)
        \end{functional}
    	\item Permite conocer los tiempos en cada ubicación de la trayectoria.
        
      \item Permite hacer consultas filtradas para el usuario mediante un formulario.
      
			\item La lógica de la aplicación debe estar programada en Python 3.
			
      \item Se mostrará en una interfaz web programada en Javascript.

    \item La base de datos usada será PostgreSQL.        
\end{functional}

\paragraph{Requisitos no funcionales}

\begin{nonfunctional}
        \item El programa está aislado de internet.
        \begin{nonfunctional}
                \item La aplicación funciona correctamente en local.
                \item Puede hacer consultas a Internet, siempre y cuando estas consultas no impliquen el envío de datos policiales.
        \end{nonfunctional}
\end{nonfunctional}