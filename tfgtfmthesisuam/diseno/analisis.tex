\paragraph{Requisitos funcionales}

\begin{functional}
        \item Permite definir el formato de los datos mediante una interfaz gráfica.
        \begin{functional}
                \item Este es el primer punto anidado.
        \end{functional}
        \item Permite  conocer las trayectorias de los terminales (mediante un algoritmo).
        \item descarta las ubicaciones irrelevantes
        \item Permite hacer consultas filtradas para el usuario.
\end{functional}

\paragraph{Requisitos no funcionales}

\begin{nonfunctional}
        \item El programa está aislado de internet.
        \begin{nonfunctional}
                \item Este es el primer punto anidado.
        \end{nonfunctional}
        \item El programa tiene una interfaz gráfica
        \item La lógica de la aplicación debe estar programada en Python.
        \item Se muestran las trayectorias en un mapa.
\end{nonfunctional}