\section{Diseño\label{SEC:DISENO}}
  Tras la evaluación de las diferentes herramientas en el estado del arte, aquí asignamos las que usaremos finalmente, junto al criterio que hemos seguido a la hora de hacerlo.
  
  La BD que usaremos será PostgreSQL. Esto es debido a que para esta aplicación, se ha observado que no existe mucha complejidad a la hora de almacenar los datos.
  
  Debido a que la envergadura del proyecto no es muy grande, el framework que se utilizará será Flask, pues es más sencillo que Django, y proporciona prácticamente las mismas herramientas. Por lo que en este caso primaremos la simpleza, para que el proyecto no sea innecesariamente complicado.
  
  Finalmente se ha decidido usar la \dfn{api} de Google Maps\cite{gmaps} a pesar de que su uso conlleve un gasto mensual. Esto es porque, tras probar las diferentes alternativas, se ha comprobado que esta \dfn{api} es la única que tiene un módulo de Roads para detectar carreteras cercanas que funciona aceptablemente y que tiene un soporte y una documentación más cuidada.
  %Diagrama casos uso
  %Diagramas para explicar codigo