\section{Dificultades Encontradas\label{SEC:DIFICULTAD}}
  \subsection{API de mapas}
    Durante la implementación del módulo de mapas, se optó en un principio por tratar de usar librerías y maptiles opensource. Más concretamente, usamos la librería de Leaflet junto a los maptiles proporcionados por OpenStreetMap.
    El problema surgió cuando comprobamos que con esta API contenía una funcionalidad muy rudimentaria y poco usada para la búsqueda de carreteras cercanas a un punto en el mapa. Esta funcionalidad está provista por la API de overpass de OpenStreetMap. Esta API resulta ser rudimentaria 
    Por ese motivo, se tuvo que buscar una alternativa que tuviera esta funcionalidad. Esa alternativa acabó siendo la API de Google Maps, que como explicamos, es de pago, pero es la más completa y usada del mercado.
    Tiene sus desventajas, pues no permite descargarse los mapas offline, y tiene que mantenerse conectado a la red para cargar los maptiles.
    %overpass
    %datos