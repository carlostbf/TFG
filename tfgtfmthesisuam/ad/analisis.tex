\section{Análisis de Requisitos\label{SEC:ANALISIS}}
  A continuación se detallarán los requisitos funcionales y no funcionales que tiene que cumplir la aplicación.
  \paragraph{Requisitos funcionales}
  
  \begin{functional}
    \item Permitir definir el formato de los datos mediante un fichero JSON.
    \item Permitir insertar los datos desde un fichero excel con el formato definido en el fichero JSON.
    
    \item Permitir filtrar los terminales en una ubicación del mapa y los momentos en los que han estado allí, dados:
    \begin{functional}
      \item sus coordenadas en latitud y longitud.
      \item un radio de búsqueda.
      \item un intervalo de tiempo entre 2 fechas.
    \end{functional} 
    \item Permitir filtrar las ubicaciones, los instantes de tiempo y la trayectoria seguida por un teléfono objetivo dados:
    \begin{functional}
      \item El número de teléfono objetivo.
      \item un intervalo de tiempo entre 2 fechas.
    \end{functional} 
    \item Se descartarán las ubicaciones irrelevantes. (A concretar!)!! %Filtro de tiempo
    \item Tener una interfaz gráfica compuesta por:
    \begin{functional}
      \item Un mapa interactivo para visualizar los resultados
      \item Un formulario web para realizar los distintos filtros de búsqueda.
    \end{functional}
  \end{functional}
  
  \paragraph{Requisitos no funcionales}
  
  \begin{nonfunctional}
    \item La lógica de la aplicación debe estar programada en Python 3.
    \item Se mostrará en una interfaz web programada en Javascript.
    
    \item La aplicación está aislada de internet.
    \begin{nonfunctional}
      \item La aplicación funciona correctamente en local.
      \item Pero permite realizar consultas en Internet, siempre y cuando estas no impliquen el envío de datos policiales.
    \end{nonfunctional}
  \end{nonfunctional}